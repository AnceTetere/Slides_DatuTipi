\documentclass{beamer}
\usepackage{xeCJK}
\usepackage{enumitem}
\usepackage{animate}

\definecolor{hrefcol}{RGB}{0, 0, 255} % Example: blue color

% meta-data
\title{SQL DATU TIPI}
%\subtitle{Using \LaTeX\ to prepare slides}
\author{\href{mailto:Ance.Tetere@csp.gov.lv}{Ance Tetere}}
\date{16. maijs, 2025}

% document body
\begin{document}

    \maketitle

    \begin{frame}
            \vspace{-1cm}
            \centering Kura ir visgrūtākā programmēšanas valoda? \\ \pause
            \vspace{5mm}
            \centering \color[RGB]{112, 0, 0}{Pirmā programmēšanas valoda.}
    \end{frame}

 %   \begin{frame}
 %       This template is a secondary creation of \bhref{https://www.overleaf.com/latex/templates/sintef-presentation/jhbhdffczpnx}{SINTEF Presentation} template from \bhref{mailto:federico.zenith@sintef.no}{Federico Zenith} \vspace{\baselineskip}

 %       Following is a brief introduction written by \bhref{mailto:federico.zenith@sintef.no}{Federico Zenith} about how to use \LaTeX\ and beamer to prepare slides. All rights reserved by him\vspace{\baselineskip}

%        This template is released under \bhref{https://creativecommons.org/licenses/by-nc/4.0/legalcode}{Creative Commons CC BY 4.0} license
%    \end{frame}

    \section{Programmēšana valoda}

    \begin{frame}
    \centering
    \vspace{-2cm}
    \includegraphics[width=5cm]{gallery/R_book.jpg}
    \end{frame}

    \begin{frame}{Programmēšana valoda}
    \vspace{-2.5cm}
    \hspace{11.5cm}
    \includegraphics[width=2cm]{gallery/R_book.jpg}\\
\vspace{0.5cm}
\hspace{1cm}
    "Šajā projektā jūs esat arī iemācījušies runāt ar datoru. \pause 
    SQL ir valoda \\ \vspace{0.1cm} \hspace{1cm} tāpat kā angļu, spāņu vai vācu, izņemot to, ka tajā runā ar mašīnām.\\ \pause \vspace{0.1cm}
    \hspace{1cm} Jūs esat iepazinušies ar šīs valodas lietvārdiem - objektiem. \pause Un, cerams, \\ \vspace{0.1cm} \hspace{1cm} ka uzminējāt, ka funkcijas ir darbības vārdi. \pause Vai argumenti funkcijā var- \\ \vspace{0.1cm} \hspace{1cm} ētu būtu apstākļa vārdi? 
    \end{frame}

    \begin{frame}{Programmēšana valoda}
    \vspace{-1.5cm}
    \hspace{11.5cm}
    \includegraphics[width=2cm]{gallery/R_book.jpg}\\
\vspace{-0.5cm}
\hspace{1cm}
    Darbības vārdi kopā ar lietvārdiem veido teikumu. \\ \pause 
    \vspace{0.4cm}
    \hspace{1cm} 
    Apvienojot funkcijas ar objektiem, jūs izsakāt veselu domu. \\ \pause 
    \vspace{0.4cm}
    \hspace{1cm} 
    Kārtojot domas loģiskā secībā, jūs attīstāt argumentu. \\ \pause
    \vspace{0.4cm}
    \hspace{1cm}
     \onslide<1->{Attīstot argumentu SQL, jūs rakstāt elegantu kodu."} \\ 
    \vspace{0.4cm}
    \hspace{6cm} (pārfrāzēts no \bhref{https://rstudio-education.github.io/hopr/}{Grolemund, 2014}, 34.\,lpp.) \\
    \end{frame}

\section{SQL valoda}
  \begin{frame}{SQL valoda}
\vspace{-0.3cm}
    \centering 
    \color[RGB]{112, 0, 0}{\textbf{SQL ir programmēšanas valoda.}} \\ \pause
    \vspace{0.4cm}
    \color[RGB]{1, 0, 92}{\textbf{Structured Query Language.}}\\ \pause
    \vspace{0.4cm} 
        \color[RGB]{112, 0, 0}{\textbf{Strukturēta vaicājumu valoda.}}
    \end{frame}

    \begin{frame}{SQL valoda}
      \vspace{-0.5cm}
      \hspace{1cm}
        SQL ir strukturēta vaicājumu valoda:
        \vspace{0.3cm}
        {\setlist[itemize]{left=1.3cm}
        \begin{itemize}
         \item[\textbullet] SQL lietvārdi: datubāzes tabulas, atribūti, shēmas.
        \item[\textbullet] SQL darbības vārdi: \texttt{LEN()}, \texttt{COUNT()}, \texttt{DATEDIFF()}.
 \end{itemize}}
    \end{frame}


    \begin{frame}{SQL valoda}
      \vspace{-0.5cm}
      \hspace{1cm}
        {\setlist[itemize]{left=1.3cm}
        \begin{itemize}
         \item[\textbullet] SQL lietvārdi: datubāzes tabulas, atribūti, shēmas.\\ \vspace{0.5cm}
        \hspace{2cm} \emph{Datubāzes objekti}: 
        %\\ \vspace{0.1cm} \hspace{2cm} 
        SQL darbojas ar ierakstu kopām. \\ \pause \vspace{0.5cm}
        \item[\textbullet] SQL darbības vārdi: \texttt{LEN()}, \texttt{COUNT()}, \texttt{DATEDIFF()}. \\ \vspace{0.5cm}
        \hspace{2cm} \emph{T-SQL}: 
        %\\ \vspace{0.1cm} \hspace{2cm} 
        Funkcijas nodrošina procedūras ar tiem. \\ 
 \end{itemize}}
 \end{frame}
 
  \begin{frame}{SQL valoda}
  \centering 
  \vspace{-0.3cm}
    \color[RGB]{112, 0, 0}{\textbf{SQL ir deklaratīva programmēšanas valoda.}} \\ \pause 
    \vspace{0.2cm}
    \color[RGB]{112, 0, 0}{\textbf{T-SQL ir procedurāls programmēšanas kods.}}
    \end{frame}


\section{SQL datu tipi}
       
    
    
   
 
% (T-SQL SSMS vidē).

%SQL ir deklaratīva programmēšanas valoda.
%T-SQL ir procedurāla programmēšanas valoda.



        
%        \item Produces a \texttt{pdf}: no problems with fonts, formulas, program versions
%            \item Easier to keep consistent style, fonts, highlighting, etc.
%            \item Math typesetting in \TeX\ is the best:
%            \begin{equation*}
%                \mathrm{i}\,\hslash\frac{\partial}{\partial t} \Psi(\mathbf{r},t) =
%            -\frac{\hslash^2}{2\,m}\nabla^2\Psi(\mathbf{r},t)
%            + V(\mathbf{r})\Psi(\mathbf{r},t)
%            \end{equation*}


    \begin{frame}[fragile]{Datu tipi}
        \vspace{0.5cm}
        Tabula \texttt{\color[RGB]{1, 0, 92}{Uznemumu\_darbinieki}} nes informāciju par darbinieku skaitu uzņēmumos.
        \begin{block}{Darbinieku skaits uzņēmumos}
            \begin{lstlisting}[language=SQL]
CREATE TABLE Uznemumu_darbinieki (
    uznemuma_nosaukums_txt varchar(4),
    darbinieku_skaits_num int,
    pazime char(4)
)\end{lstlisting}
        \end{block}
    \end{frame}

        \begin{frame}[fragile]{Datu tipi}
        \vspace{0.5cm}
        Tabula \texttt{\color[RGB]{1, 0, 92}{Uznemumu\_darbinieki}} nes sekojošas vērtības.
        \vspace{0.5cm}
        \begin{block}{Darbinieku skaits uzņēmumos}
            \begin{lstlisting}[language=SQL]
INSERT INTO Uznemumu_darbinieki (uznemuma_id, uznemuma_nosaukums_txt, darbinieku_skaits_num)
VALUES (1, 'UZN1', 10), 
       (2, 'UZN2', 25), 
       (3, 'UZN3', 0)
    \end{lstlisting}
    \end{block}
    \end{frame}

    
        \begin{frame}[fragile]{Datu tipi}
        \vspace{0.5cm}
        Tabula \texttt{\color[RGB]{1, 0, 92}{Uznemumu\_darbinieki}} nes sekojošas vērtības.
        \vspace{0.5cm}
        \begin{block}{Darbinieku skaits uzņēmumos}
            \begin{lstlisting}[language=SQL]
INSERT INTO Uznemumu_darbinieki (uznemuma_id, uznemuma_nosaukums_txt, darbinieku_skaits_num)
VALUES (1, 'UZN1', 10), 
       (2, 'UZN2', 25), 
       (3, 'UZN3', 0)
    \end{lstlisting}
    \end{block}
    \end{frame}

    

    \begin{frame}[fragile]{Datu tipi}
   \vspace*{0.5cm}
    Tabula \texttt{\color[RGB]{1, 0, 92}{Uznemumu\_darbinieki}}.\\
    \vspace*{0.3cm}
    \includegraphics[width=13cm]{gallery/uznemumu_darbinieki.png}
    \end{frame}

        \begin{frame}[fragile]{Datu tipi}
   \vspace*{0.5cm}
    Tabula \texttt{\color[RGB]{1, 0, 92}{Uznemumu\_darbinieki}}.\\
    \vspace*{0.3cm}
    \includegraphics[width=12cm]{gallery/uznemumu_darbinieki1.png}
    \end{frame}

        \begin{frame}[fragile]{Datu tipi}
   \vspace*{0.5cm}
   \hspace*{1cm}
    Tabula \texttt{\color[RGB]{1, 0, 92}{Uznemumu\_darbinieki}}.\\
    \vspace*{0.3cm}
    \hspace*{2cm}
    \includegraphics[width=5cm]{gallery/VARCHAR.png}
    \end{frame}

   \begin{frame}[fragile]{Datu tipi}
   \vspace*{0.5cm}
   \hspace*{1cm}
    Tabula \texttt{\color[RGB]{1, 0, 92}{Uznemumu\_darbinieki}}.\\
    \vspace*{0.3cm}
    \hspace*{2cm}
    \includegraphics[width=5cm]{gallery/INT.png}
    \end{frame}

       \begin{frame}[fragile]{Datu tipi}
   \vspace*{0.5cm}
   \hspace*{1cm}
    Tabula \texttt{\color[RGB]{1, 0, 92}{Uznemumu\_darbinieki}}.\\
    \vspace*{0.3cm}
    \hspace*{2cm}
    \includegraphics[width=5cm]{gallery/CHAR.png}
    \end{frame}

    \begin{frame}[fragile]{Datu tipi}
     \hspace*{1cm}
     \vspace*{0.3cm}
    \includegraphics[width=8cm]{gallery/uznemumu_darbinieki3.png} \pause
    \vspace*{0.3cm}
    \hspace*{4cm}
    \includegraphics[width=3cm]{gallery/uznemumu_darbinieki4.png}\pause
     \vspace*{0.3cm}
     \hspace*{1cm}
    \includegraphics[width=10cm]{gallery/uznemumu_darbinieki5.png}
    \end{frame}

        \begin{frame}[fragile]{Datu tipi}
     \hspace*{1cm}
     \vspace*{0.3cm}
    \includegraphics[width=8cm]{gallery/vilciens ar cisternām.png} 
     \vspace*{0.3cm}
     \hspace*{1cm}
    \includegraphics[width=10cm]{gallery/uznemumu_darbinieki5.png}
    \end{frame}

            \begin{frame}[fragile]{Datu tipi}
     \hspace*{1cm}
     \vspace*{0.3cm}
    \includegraphics[width=10cm]{gallery/citernas ar datiem.png}
    \end{frame}

                \begin{frame}[fragile]{Datu tipi}
     \hspace*{1cm}
     \vspace*{0.3cm}
    \includegraphics[width=10cm]{gallery/citernas ar datiem2.png}
    \end{frame}

    \begin{frame}[fragile]{Datu tipi}
     \hspace*{1cm}
     \vspace*{0.3cm}
    \includegraphics[width=10cm]{gallery/citernas ar datiem3.png}
    \end{frame}

        \begin{frame}[fragile]{Datu tipi}
     \hspace*{1cm}
     \vspace*{0.3cm}
    \includegraphics[width=10cm]{gallery/citernas ar datiem4.png}
    \end{frame}

    \begin{frame}[fragile]{Datu tipi}
    \vspace{0.5cm}
    Tabulā \texttt{\color[RGB]{1, 0, 92}{Uznemumu\_darbinieki}} tiek veikti labojumi.
        \vspace{0.5cm}
        \begin{block}{Darbinieku skaits uzņēmumos}
            \begin{lstlisting}[language=SQL]
UPDATE Uznemumu_darbinieki
   SET darbinieku_skaits_num = NULL
   WHERE darbinieku_skaits_num = 0
    \end{lstlisting}
    \end{block}
    \end{frame}

    \begin{frame}[fragile]{Datu tipi}
   \vspace*{0.5cm}
    Tabula \texttt{\color[RGB]{1, 0, 92}{Uznemumu\_darbinieki}}.\\
    \vspace*{0.3cm}
    \includegraphics[width=13cm]{gallery/TAB1.png}
    \end{frame}
    
    \begin{frame}[fragile]{Datu tipi}
    \includegraphics[width=13cm]{gallery/TAB2.png}
    \end{frame}

     \begin{frame}[fragile]{Datu tipi}
     \hspace*{1cm}
     \vspace*{-1.5cm}
    \includegraphics[width=10cm]{gallery/cisternas2_ar_datiem.png} 
     \vspace*{0.3cm}
     \hspace*{1cm}
    \includegraphics[width=10cm]{gallery/TAB4.png}
    \end{frame}


                \begin{frame}[fragile]{Datu tipi}
     \hspace*{1cm}
     \vspace*{0.3cm}
    \includegraphics[width=10cm]{gallery/cisternas2_ar_datiem1.png}
    \end{frame}

    \begin{frame}[fragile]{Title page}
        To set a typical title page, you call some commands in the preamble:
        \begin{block}{The Commands for the Title Page}
            \begin{lstlisting}[language=TeX]
\title{Sample Title}
\subtitle{Sample subtitle}
\author{First Author, Second Author}
\date{Defaults to today's}\end{lstlisting}
        \end{block}
        You can then write out the title page with \verb|\maketitle|.

        You can set a different background image than the default one with the \verb|\titlebackground| command, set before \verb|\maketitle|.

        In the \texttt{backgrounds} folder, you can find a lot of standard backgrounds for SINTEF presentation title pages.
    \end{frame}

    \begin{frame}[fragile]{Writing a Simple Slide}
        \framesubtitle{It's really easy!}
        \begin{itemize}[<+->]
            \item A typical slide has bulleted lists
            \item These can be uncovered in sequence
        \end{itemize}
        \begin{block}{Code for a Page with an Itemised List}<+->
            \begin{lstlisting}[language=TeX]
\begin{frame}
    \frametitle{Writing a Simple Slide}
    \framesubtitle{It's really easy!}
    \begin{itemize}[<+->]
        \item A typical slide has bulleted lists
        \item These can be uncovered in sequence
    \end{itemize}
\end{frame}\end{lstlisting}
        \end{block}
    \end{frame}

    \begin{frame}[fragile]{Adding images}
        Adding images works like in normal \LaTeX:
        \begin{columns}
            \begin{column}{0.7\textwidth}
                \begin{block}{Code for Adding Images}
                    \begin{lstlisting}[language=TeX]
\usepackage{graphicx}
% ...
\includegraphics
[width=\textwidth]{gallery/img.jpg}
                    \end{lstlisting}
                \end{block}
            \end{column}
            \begin{column}{0.3\textwidth}
                \vskip5pt\includegraphics[width=\textwidth]{gallery/img.jpg}
            \end{column}
        \end{columns}
    \end{frame}

    \begin{frame}[fragile]{Splitting in Columns}
        Splitting the page is easy and common; typically, one side has a picture and the other text:
        \begin{columns}
            \begin{column}{0.6\textwidth}
                This is the first column
            \end{column}
            \begin{column}{0.3\textwidth}
                And this the second
            \end{column}
        \end{columns}
        \begin{block}{Column Code}
            \begin{lstlisting}[language=TeX]
\begin{columns}
        \begin{column}{0.6\textwidth}
                This is the first column
        \end{column}
        \begin{column}{0.3\textwidth}
                And this the second
        \end{column}
        % There could be more!
\end{columns}\end{lstlisting}
        \end{block}
    \end{frame}

    \begin{frame}[fragile]
        \frametitle{Fonts}
        \begin{itemize}
            \item The paramount task of fonts is being readable
            \item There are good ones...
                \begin{itemize}
                \item {\textrm{Use serif fonts only with high-definition projectors}}
                \item {\textsf{Use sans-serif fonts otherwise (or if you simply prefer them)}}
                \end{itemize}
            \item ... and not so good ones:
                \begin{itemize}
                \item {\texttt{Never use monospace for normal text}}
                \item {\frakfamily Gothic, calligraphic or weird fonts: should always: be
                avoided}
            \end{itemize}
        \end{itemize}
    \end{frame}

    \begin{frame}[fragile]{Look}
        \begin{itemize}
        \item To change the colour of the title dash, give one of the class options \texttt{cyandash} (default), \texttt{greendash}, \texttt{magentadash}, \texttt{yellowdash}, or \texttt{nodash}.
        \item To change between the light and dark themes, give the class options \texttt{light} (default) or \texttt{dark}. It is not possible to switch theme for one slide because of the design of Beamer---and it's probably a good thing.
        \item To insert a final slide, use \verb|\backmatter|.
        \item The aspect ratio defaults to 16:9, but you can change it to 4:3 for old projectors by passing the class option \texttt{aspectratio=43}; any other values accepted by Beamer are also possible.
        \end{itemize}
    \end{frame}

    \section{Summary}

    \begin{frame}
        \frametitle{Good Luck!}
        \begin{itemize}
            \item Enough for an introduction! You should know enough by now
            \item If you have corrections or suggestions, \bhref{mailto:federico.zenith@sintef.no}{send them to me!}
        \end{itemize}
    \end{frame}

    \QApage

\end{document}
